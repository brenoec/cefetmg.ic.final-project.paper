
\section{Conclusão}

É frequente o uso de redes neurais artificiais nos casos em que se tem pouca
informação acerca de um determinado problema e não se tem meios de estabelecer
uma solução analítica para o mesmo.

A rede neural artificial proposta apresentou tempo de treinamento factível para
aplicações reais ($10$ segundos) e apresentou desempenho igualmente factível
($600$ amostras em aproximadamente $1$ segundo), passível de ser implantada em
sistemas embarcados e de fazer classificações em tempo real. Todas as amostras
foram classificadas corretamente pela rede proposta.
