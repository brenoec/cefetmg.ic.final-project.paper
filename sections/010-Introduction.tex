
\section{Introdução}

\lettrine{O}{} reconhecimento de locutor por meio automático é um campo de estudo
tecnológico que desde o surgimento das maquinas, sempre fez parte das aspirações
humanas. Isto se deve às facilidades inerentes a sistemas capazes de obter êxito
nessa tarefa, tais como fazer com que um computador identifique seu dono através
da fala, o controle de e robôs e maquinas em uma linha de montagem, sem a
necessidade de entradas codificadas ou linguagens específicas e de difícil
acesso aos seres humanos.

As redes neurais artificiais (RNAs) são aliadas neste sentido, pois através
delas é possível implementar modelos de reconhecimento de locutor e tornar
viável inúmeras outras comodidades resultantes deste tipo de aplicação. Existem
alguns registros na literatura sobre aplicações para as técnicas de
reconhecimento de locutor como, por exemplo, o auxílio a portadores de
deficiência \cite{Neto:2010:ReconhecimentoVozCadeiraRodas}
\cite{Thiang:2007:SpeechRecognitionWheelchair}, aplicações na prática forense
\cite{Cardoso:2001:InteligenciaArtificialJudiciario} e automação de ambientes
\cite{Amaral:2004:ReconhecimentoVozAutomaçãoInteligente}.

Por outro lado, verifica-se a inexistência de grande diversidade na literatura
da área, visto que os sistemas que permitem uma comunicação mais natural entre
homem e máquina ainda não estejam plenamente dominados
\cite{Neto:2010:ReconhecimentoVozCadeiraRodas}.

Esse trabalho busca demonstrar através de uma abordagem prática, a aplicação de
reconhecimento de locutor através de uma rede projetada e analisada para este
fim. Para isto, serão selecionados pequenos trechos de áudios tratados e
encaminhados à rede para que sejam classificados pela mesma.

A rede neural artificial em questão deve ser capaz de reconhecer a voz de
determinado locutor dentre um numero de locutores aleatórios. Ruídos serão
incrementados aos áudios a fim de oferecer maior poder de generalização. A soma
ruído e áudio passa pelo mesmo processo de tratamento que o áudio original.

O restante do artigo está organizado de forma direta, com metodologia,
resultados, ameaças à validade e conclusão.

A seção 2 aborda com detalhes a metodologia de trabalho, com ênfase na aquisição
e manipulação dos dados de entrada. Se expõe como que informações referentes aos
áudios de entrada são utilizadas como entrada da rede projetada. A arquitetura
da rede é brevemente discutida.
