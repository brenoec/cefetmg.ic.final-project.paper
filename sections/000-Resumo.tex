
\renewcommand{\abstractname}{Resumo}
\begin{abstract}
\noindent Breve discussão sobre reconhecimento de locutor. Abordagem via rede neural
artificial para reconhecimento de locutor através de trechos de curta duração de
áudio de entrada. Sinal de entrada é tratado com finalidade de remover dados não
pertinentes, bem como reduzir a complexidade da rede neural que fará a
classificação. A entrada da rede neural é o espectrograma do sinal de entrada,
adquirido através da transformada de Fourier do áudio tratado. A rede estudada
mapeia subespaços não lineares e descontínuos dos possíveis valores de entrada.
O desenvolvimento da rede, seu desempenho e possível aplicabilidade em cenários
reais são avaliados. O tempo de treinamento da rede analisada é da ordem de
10 segundos, sendo que a classificação de todas as amostras ocorre em tempo
inferior a 1 segundo. A rede neural avaliada classificou corretamente todas as
amostras de áudios de entrada.

\hfill \break
\hfill \break
\textbf{Palavras-chave:} reconhecimento de locutor, redes neurais artificiais, classificação.
\end{abstract}
