
\usepackage[backend=biber,sorting=none]{biblatex}
\renewbibmacro{in:}{}
\setlength\bibitemsep{1.5\itemsep}
\renewcommand*{\bibfont}{\footnotesize}
\addbibresource{refbase.bib}

\usepackage{lipsum} % Package to generate dummy text throughout this template

\usepackage{csquotes} % babel brazil dependency
\usepackage[brazil]{babel} % pt-BR silabic separations
\usepackage[sc]{mathpazo} % Use the Palatino font
\usepackage[fleqn]{amsmath} % math symbols
\usepackage{lmodern} % fonts
\usepackage{pifont} % fonts
\linespread{1.05} % Line spacing - Palatino needs more space between lines
\usepackage{microtype} % Slightly tweak font spacing for aesthetics

\usepackage[hmarginratio=1:1,top=32mm,columnsep=20pt]{geometry} % Document margins
\usepackage{multicol} % Used for the two-column layout of the document
\usepackage[small,labelfont=bf,up,textfont=it,up]{caption} % Custom captions under/above floats in tables or figures
\usepackage{booktabs} % Horizontal rules in tables
\usepackage{float} % Required for tables and figures in the multi-column environment - they need to be placed in specific locations with the [H] (e.g. \begin{table}[H])
\usepackage{hyperref} % For hyperlinks in the PDF
\usepackage{graphicx}

\usepackage{lettrine} % The lettrine is the first enlarged letter at the beginning of the text
\usepackage{GoudyIn}
\usepackage[x11names]{xcolor}
\renewcommand{\LettrineFontHook}{\GoudyInfamily{}}
\LettrineTextFont{\itshape}
\setcounter{DefaultLines}{3}

\usepackage{paralist} % Used for the compactitem environment which makes bullet points with less space between them

\usepackage{abstract} % Allows abstract customization
\renewcommand{\abstractnamefont}{\normalfont\bfseries} % Set the "Abstract" text to bold
\renewcommand{\abstracttextfont}{\normalfont\small\itshape} % Set the abstract itself to small italic text

\usepackage{titlesec} % Allows customization of titles
\renewcommand\thesection{\Roman{section}} % Roman numerals for the sections
\renewcommand\thesubsection{\Roman{subsection}} % Roman numerals for subsections
\titleformat{\section}[block]{\large\scshape\centering}{\thesection.}{1em}{} % Change the look of the section titles
\titleformat{\subsection}[block]{\large}{\thesubsection.}{1em}{} % Change the look of the section titles

\usepackage{nameref} % ref sections on text
\makeatletter
\newcommand*{\currentname}{\@currentlabelname}
\makeatother

\usepackage{fancyhdr} % Headers and footers
\pagestyle{fancy} % All pages have headers and footers
\fancyhead{} % Blank out the default header
\fancyfoot{} % Blank out the default footer
\fancyhead[C]{Reconhecimento de Locutor via Rede Neural Artificial {\footnotesize \ding{118}} \currentname} % Custom header text
\fancyfoot[RO,LE]{\thepage} % Custom footer text

\renewcommand{\figurename}{Figura}
\renewcommand{\tablename}{Tabela}
